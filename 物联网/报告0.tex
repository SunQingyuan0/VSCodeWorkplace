\documentclass[UTF8]{ctexart}
\usepackage{amsmath}
\usepackage{graphicx}
\usepackage{ctex}
\usepackage{graphicx}
\usepackage{subfigure}
\usepackage{float}
\usepackage{listings}
\usepackage{fancyhdr}
\usepackage[comma,square,super]{natbib}
\usepackage{booktabs}
\usepackage[framemethod=TikZ]{mdframed}
\usepackage{url}   % 网页链接
\usepackage[comma,square,super]{natbib}
\usepackage{cleveref}
\usepackage{amsmath}
\usepackage{multirow}
\usepackage{longtable}
\usepackage{graphicx}
\usepackage{array}
\usepackage{supertabular}
\CTEXsetup[format={\Large\bfseries}]{section}
\title{\textbf{报告0.0}}
\pagestyle{fancy}
\fancyhead{}
\renewcommand\headrulewidth{0pt}
\begin{document}
\date{}
\maketitle
\section{功能设想}
\begin{itemize}
    \item 道路中出现较近障碍物时,产品发送讯息至手机告知障碍物的距离信息,及时提醒避开障碍物
    \item 所在区域交通信息网络联合。每当行经十字路口、交叉路口时,会告知此时的红绿灯状况
    \item  内置微波传感器判判断道路中速度较快的物体,并通过app发出警报警示。
    \item 内置 GPS 确定每时每刻的方位
    \item 遭遇危险或身体不适可发送讯息至亲人
\end{itemize}

《基于可穿戴的智能盲人导航设计》李诗芸

\section{设计模块}
\subsection{环境检测}
包括:四周障碍物大小与方位,地面凹陷情况,楼梯及电梯位置,其他行人方
位,室内设施位置,室外斑马线、交通灯、交通指示牌等交通标志信息。
\subsection{人机交互}
人机界面的简单化,不是纯净,刻意简单的几何造型,而是去除各种不必要的
装饰,去除各种干扰人们使用的因素,使得产品技术简单,功能明确,言简意赅的
设计。简单的人机界面,更易于用户理解和记忆

同时也要考虑雷达,云端,服务器与手机之间的逻辑
《导盲机器人研究现状综述》武曌晗
\section{原则}
\begin{itemize}
    \item 保障不同身体尺寸、姿势和行动能力的人群在不同状态下的安全性和便携性
    \item 以多元化的方式传达必要的信息。考虑到不同群体的使用习惯和操作方式可以选用检测或扫描等功能,以及语音或震动等提醒方式
    \item 在不同的体格,姿势,移动能力下,提供轻松的操作空间。
    \item 适应不同情境的操作需求。在检测到障碍物时,会根据障碍物危险程度的不同,以不同的震动频率或者语音提醒。
\end{itemize}

《盲人助行产品的无障碍设计研究》尚琳琳
\end{document}