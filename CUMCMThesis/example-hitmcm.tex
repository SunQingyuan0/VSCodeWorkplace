% !Mode:: "TeX:UTF-8"
% !TEX program  = xelatex

%% ===+++--------------------
%% TeX注释: % 行注释
%% 转义字符: `\`, 如, \%, \_ 等
%% 保留词: # $ % ^ & _ { } ~ \
%% 引号: ``引号内文字''
%% 建议使用TeXLive或TeXstudio编辑
%% 注意, 能通过修改.tex实现的, 就不要修改`cumcmthesis.cls`文件
%% 文中用 % TODO 标记的部分, 可能需要修改, 需要特别注意.
%% -- hitmcm@gmail.com, 2018.04-
%% ===+++--------------------

%\documentclass{cumcmthesis}

\documentclass[withoutpreface,bwprint]{cumcmthesis} %去掉封面与编号页,电子版提交的时候使用。
\usepackage{booktabs}
\usepackage[framemethod=TikZ]{mdframed}
\usepackage{url}   % 网页链接
\usepackage{subcaption} % 子标题
\usepackage[comma,square,super]{natbib}
\usepackage{cleveref}
\usepackage{amsmath}
\usepackage{multirow}
\usepackage{longtable}
\usepackage{graphicx}
\usepackage{array}
\usepackage{supertabular}
% TODO 导言区, 便于扩展
% 自定义源代码样式
%\lstset{%
%  numbers=left,  % 行号
%  numberstyle={\color{gray}\tiny},  %
%  frame=shadowbox,  % 边框
%}

\title{生产企业原材料的订购与运输}  % TODO 论文标题
%\tihao{A}
%\baominghao{4321}
%\schoolname{XX大学}
%\membera{}
%\memberb{}
%\memberc{}
%\supervisor{}
%\yearinput{2020}
%\monthinput{08}
%\dayinput{22}

\begin{document}

\maketitle
\begin{abstract}  % TODO 摘要, 非常重要!!!
企业生产过程中,原材料的供应十分重要,这直接影响了一个企业的生产活动。但是供货商提供的货并不一定会严格按照企业所下的订单执行,经常多一些或少一些,甚至无法交付。本文通过分析某企业进五年来,与各个供货商之前的商业来往,及各个运输企业的五年运输损耗数据,通过建立数学模型的方式,为企业制定一个尽量好的订购方式,以满足企业生产所需。

对于问题一,根据题目,使用五年来,与各个供货商之前的订货量和供货量数据,尝试依照这些数据,制定供货总量,供货均值,稳定性,运营稳定度,供货准确性,大额订单比例,最后使用一个公式,此公式要经过不断的调优,将他们核算为一个标准评分,根据评分为各个公司排名。

对于问题二,要使用最少的供货商数目,采用数学规划模型,由题意,企业产能为2.82万立方米,且库存必须大于两周,故每周开始时企业所有的原料要大于2.82万立方米的两倍,并且第一周要购入两周的原料。这为本题的数学规划模型的第一条约束条件。为下面计算,还要估算每家供货商能提供的最大供货量,根据前五年的供货数据,使用一个时间序列,预测接下来24个月各个供货商的最大供货量。企业所下的订货量,经加入一个随机变量,用来估算供货量,这个供货量要小于这家供货商的最大供货量。由此两个约束条件,对最少供货商这一目标函数进行优化,可以得到所需的供货商名单。根据这些供货商名单,采用同样的约束条件,对原料成本这个目标函数进行优化,可得最经济的订购方案,再根据这个订购方案,可以计算出转运方案,同样采用数学规划的方法,约束条件为每家转运公司每天最大转运量为6000立方米,和每家供货商最多只能有一家转运商。

对于问题三和问题四,方法与第二题相同,不用求供应商名单,此处的企业名单为全部的402家供应商,目标函数分别换为仓储成本,最小损耗和最大接收量即可。
% 电子版论文第一页为摘要专用页(含标题和关键词,但不需要翻译成英文),从此页开始编写页码;页码必须位于每页页脚中部,用阿拉伯数字从“1”开始连续编号。\textbf{\color{red}摘要专用页必须单独一页,且篇幅不能超过一页}。
% 论文正文(\textbf{\color{red}不要目录},尽量控制在20页以内);正文之后是论文附录(页数不限)。

\keywords{数学规划;时间序列;多目标优化;}  % TODO 关键词, 一般3~5个, 按词条的外延层次从大到小排列
\end{abstract}

% TODO 自动生成目录, 2019 明确不要目录
%\tableofcontents
%\newpage

\section{问题重述}

\subsection{问题背景}
近年来,市场竞争不断加剧,企业只有不断提高生产能力和自身管理水平,才能在充满变化的环境中继续生存和发展。现如今,越来越多的企业开始重视起原材料的采购和管理,现代企业管理学认为,企业的采购与运输管理管理是企业的“第三利润来源”,生产成本与企业的经济效益息息相关\cite{bib:1}。因此,降低企业的原材料采购与运输成本已经成为越来越热门的问题。

\subsection{问题的提出}

某企业订购A,B,C三种原材料,企业当前产能为2.82万$m^3$,且每0.6$m^3$的材料A,或每0.66$m^3$的材料B,或每0.72$m^3$的材料C都可以转化为$1m^3$的产品。其中,A类和B类原材料的单价分别是C类原材料的1.2和1.1倍,每家转运商的运输能力均为6000$m^3$/周。该企业每年安排48周进行生产,现有过往5年402家供应商订货量与供货量,8家转运商的运输损耗率,根据上述数据和背景建立相应数学模型求解如下问题:

(1)以保障企业生产为目标,分析过往数据,建立供应商评价指标体系,建立相应评价模型并给出50家最优的供应商。

(2)根据问题1选出尽量少的供应商来满足企业生产需求,制定未来24周的订购计划并制定转运方案,使得原材料采购和转运成本降到最低。分析方案的实施效果。

(3)企业尽量多的采购原料A,尽量少的采购C,重新制定最经济的采购和转运方案,并分析方案的实施效果。

(4)假定该企业可以提高产能,根据供应商和转运商的情况,确定企业产能可以提高多少,并制定采购和转运的方案。

\section{问题分析}
我们将研究这家建筑材料生产公司的原材料订购问题,研究内容是从已有数据中,为这家公司接下来半年的时间制定采购计划。问题附件给出了这家公司5年的采购和供货的历史数据,但两者并不完全吻合,即供货时并不会完全按照采购量供货,因此,我们要在很多不确定因素下,为该公司进行采购方案决策。
\subsection{问题一的分析}
采购来源的402家,提供了五年全部的订购量和供货量,这些数据显然无法直接作为评判某家供货商供货的优劣,所以我们需要将这些数据抽象为一些其他的统计数据,如平均数,方差,统计值等数据和他们之间的互相组合,依靠这些数据,我们使用直接方程或者各种评价方法,为每家企业计算出一个评价值,根据评价结果,还要对不合理的评价进行调整,调整评价模型中的各项参数或者系数,已获得一个更加良好的评价,将真正对于其企业运作有重大帮助的企业给予尽量高的分数。本题难点在于模型的选取和评价参数的抽象,这两点直接影响的评价的结果的有效性。
\subsection{问题二的分析}
(1)对于第一小问,该问题需要我们求出最少使用供货商的数目。首先,我们想要尽量少的供货商时,还要保证收到的货物满足企业生产所需,因此要供货企业尽量多的供货,我们采用了一个时间序列模型来预测接下来半年时间,供货商能最多给我们供多少货物,依此我们可以使用数学规划的方法,计算出最少需要的供货商个数。

(2)对于第二小问,针对这些供货商,其中部分供货商并没有在一些事件为企业供货,这给我们调整计划带来了可能,根据ABC三种货物的价格,建立目标函数,依旧使用数学规划的方法,可以计算出最低成本。

(3)对于第三小问,以上的供货公司的订货量已经有第二小问求出,这里便可以直接拿来计算供货量,依照这个供货量和题目中给出的各个转运公司的损失量,可以计算出每家公司的损失大概的一分布,因为转运时各周哥公司的损失量不尽相同故采取一个随机变量进行估计并进行数学规划。分析该订购方案和转运方案的效果,即将上述几个数学规划重新进行几次,看最后目标函数的改变情况用以分析效果
\subsection{问题三的分析}
本题我们要在两个目标下进行订货量和转运商的选择,即多目标的数学规划,分别对A和C两种货物给上较小和较大的权重,并为两个目标分别给予不同的权重,即可得到结果。
\subsection{问题四的分析}
本问题要求出产能的最大值,整个问题中,对产能的限制只有原料一条,即越多的原料,就能提升越多的产能,按照之前对供应商供应能力的预测,照单全收,即可达到最大产能。
\section{模型假设}

\begin{itemize}
\item 供货量围绕订货量波动;
\item 企业对一家供应商提供的原材料总是全部收购;
\item 供货商运营稳定;
\item 转运商损失率在一定范围内随机波动;
\item 一家供应商的原材料由一家转运商运输;
\item 企业库存应保障两周的生产需求。
\end{itemize}

\section{符号说明}

% 这里仅列出论文中一些重要符号及其含义, 符号在正文第一次出现时, 仍需说明符号的含义.
\begin{table}[!htbp]
\caption{符号说明}\label{tab:Nomenc} \centering  % 引用: \ref{tab:Nomenc}
\begin{tabular}{llc}  % 三线表
\toprule[1.5pt]
 符号 & 含义 & 单位 \\
\midrule[1pt]
 $w=1,2,\ldots240$  & 每周的序号 &   \\
 $s=1,2,\ldots402$  & 供应商的序号 &    \\
 $t=1,2\ldots8$ & 转运商的序号 & \\
 $D_{w,s}$  & 企业在第$w$周对某供应商$s$的订货量 & $m^3$ \\
 $G_{w,s}$  & 第$w$周某供应商$s$的供货量 & $m^3$ \\
 $D_{A,w}$  & 企业在第$w$周对于原材料$A$的总订货量   & $m^3$ \\
 $G_{A,w}$  & 第$w$周对于原材料$A$的总供货量   & $m^3$ \\
 $L_{t,w}$  & 第$w$周第$t$家转运商的损失率   &  \\
 $R_{A,w}$  & 企业在第$w$周对于原材料$A$的总收货量   & $m^3$ \\
 $\mu_s$  & 供应商$s$供应量的平均增幅   & $m^3$ \\
 $N_s(\mu,\sigma^2)$  & 供应商$s$供应量的随机波动   & $m^3$ \\
 $K_w$  & 第$w$周生产剩余的原料  & $m^3$ \\
\bottomrule[1.5pt]
\end{tabular}
\end{table}

\section{模型的建立与求解}

\subsection{问题一的模型}
本问以保障企业的生产为前提,首先分析了企业在选择供应商时的影响因素,选取了若干评价指标,并据此建立了供应商评价体系。基于此评价体系给出了供应商的评价公式,可帮助企业选择更加优质的供应商。通过建立供应商评价体系与评价公式,我们对402家供应商进行打分,并得出了最优的50家供应商。
\subsubsection{模型建立}
\leftline{\textbf{(一)供应商评价体系建立}}

有关评价供应商的指标有很多,我们参考了Dickson G.W.\cite{bib:2}与Weber C.A.的研究\cite{bib:3},在本问题中,我们从实际出发,根据附件1中所给出的供货量及订货量的数据,选取出6个指标来衡量供应商保障企业生产的能力。

\begin{itemize}
    \item \textbf{供货总量}
\end{itemize}

一家供应商的供货总量$G_s$为该供应商全部周数的供应量累加,由于不同的供应商提供的原材料种类不同,为了方便后续的计算,我们将不同原材料的体积量统一换算成了产品的体积。

\begin{equation}
G_s=\frac{\sum_{i=1}^{w}G_{w,s}}{V_a}
\label{eq:Ga}
\end{equation}

若一家供应商供应原材料A,则该供应商的供货总量如\cref{eq:Ga}计算,其中,$w$为周数,本题取240,$G_{w,s}$为第$w$周某供应商$s$的供货量,$V_a$为每立方米产品消耗的原材料A,题中取0.6。

\begin{equation}
G_s=\frac{\sum_{i=1}^{w}G_{w,s}}{V_b}
\label{eq:Gb}
\end{equation}
\begin{equation}
G_s=\frac{\sum_{i=1}^{w}G_{w,s}}{V_c}
\label{eq:Gc}
\end{equation}

同理得到供应原材料B,C的供应商的供货总量公式,分别如\cref{eq:Gb},\cref{eq:Gc}所示,其中$V_b$取0.66,$V_c$取0.72。

\begin{itemize}
    \item \textbf{供货均值}
\end{itemize}

供货均值$A_s$衡量一家供应商平均每周的供货能力。

\begin{equation}
A_s=\frac{G_s}{W_{G\neq0}}
\label{eq:As}
\end{equation}

其中$G\neq0$表示该供应商供货量不为0的周数。

\begin{itemize}
    \item \textbf{供货稳定度}
\end{itemize}

$D_s$为一家供应商的供货稳定程度,通过供货量的标准差来计算。

\begin{equation}
D_s=\log_2(S+1)
\label{eq:Ds}
\end{equation}

$S$为该公司供货量标准差。

\begin{itemize}
    \item \textbf{运营稳定度}
\end{itemize}

以一家供应商所接到订单的周数占比衡量该供应商的运营情况。

\begin{equation}
\eta=\frac{W_{D\neq0}}{W}
\label{eq:eta}
\end{equation}

如\cref{eq:eta}所示,$\eta$表示运营稳定度,$W_D\neq0$是订货量不为0的周数,$W$是总周数。

\begin{itemize}
    \item \textbf{供货准确性}
\end{itemize}

对于一家供应商,其每周的订货量与供货量之间存在一定误差,因此需要一个指标来衡量供应商供货量与订货量之间的一致程度,这里确定了供货准确性$C_s$的计算公式。

\begin{equation}
C_s=\frac{\sum_{i=1}^w\mid D_{w,s}-G_{w,s}\mid}{G_s}
\label{eq:Cs}
\end{equation}

如\cref{eq:Cs}所示,$D_{w,s}$为企业在第$w$周对某供应商$s$的订货量,$G_s$为第$w$周某供应商$s$的供货量,$G_s$为该公司供货总量。

\begin{itemize}
    \item \textbf{大额订单占比}
\end{itemize}

考虑到企业产能较大,对于本题而言,一家供应商大额订单所占比例也具有一定的参考价值。

\begin{equation}
R_s=\frac{W_{G>100}}{W}
\label{eq:Rs}
\end{equation}

$R_s$表示大额订单占比,$W_{G>100}$表示供货量大于100的周数,$W$表示总周数。

\leftline{\textbf{(二)评价模型的建立}}

上文确立的评价指标中,供货总量$G_s$,供货均值$A_s$,运营稳定度$\eta$,大额订单占比$R_s$均为极大型指标,供货稳定度$D_s$,供货准确性$C_s$为极小型指标。经查阅文献与多次试验分析得到了供应商评价指数公式。

\begin{equation}
B=\frac{(\frac{\sqrt{G_s}}{10}+A_s)(R_s+0.5)}{2}-10(1-\eta)C_sD_s
\label{eq:B}
\end{equation}
\subsubsection{模型求解}

首先通过附件1中给出的402家供应商的近5年供货量与订货量的数据计算出这些供应商的供货总量$G_s$,供货均值$A_s$,运营稳定度$\eta$,大额订单占比$R_s$,供货稳定度$D_s$,供货准确性$C_s$的值,然后带入\cref{eq:B}计算即可。
\subsubsection{评价结果}

\begin{longtable}{|l|l|l|l|l|l|l|l|}
\caption{评价结果}
\label{tab:ping} \\
\hline
        供应商ID & 供货总量 & 供货均值 & 稳定性 & 运营稳定度 & 供货准确性 & 大额订单比例 & 评分 \\ \hline
        S229 & 591478.33  & 2464.49  & 8.87  & 1.00  & 70.16  & 1.00  & 1906.05  \\ \hline
        S361 & 455666.67  & 1898.61  & 8.65  & 1.00  & 21.18  & 1.00  & 1474.59  \\ \hline
        S108 & 365075.76  & 1521.15  & 10.26  & 1.00  & 316.15  & 1.00  & 1186.18  \\ \hline
        S282 & 282233.33  & 1175.97  & 8.60  & 1.00  & 30.41  & 0.99  & 916.70  \\ \hline
        S151 & 270136.11  & 1125.57  & 10.83  & 1.00  & 739.60  & 1.00  & 883.16  \\ \hline
        S275 & 264255.00  & 1101.06  & 6.86  & 1.00  & 31.30  & 1.00  & 864.35  \\ \hline
        S329 & 260863.33  & 1086.93  & 6.91  & 1.00  & 28.00  & 1.00  & 853.50  \\ \hline
        S340 & 259736.36  & 1082.23  & 7.32  & 1.00  & 0.12  & 1.00  & 847.54  \\ \hline
        S131 & 208351.52  & 868.13  & 7.20  & 1.00  & 7.76  & 1.00  & 683.43  \\ \hline
        S308 & 207572.73  & 864.89  & 9.09  & 1.00  & 2.06  & 1.00  & 680.94  \\ \hline
        S330 & 207048.48  & 862.70  & 9.40  & 1.00  & 180.63  & 1.00  & 679.26  \\ \hline
        S268 & 180258.33  & 751.08  & 6.16  & 1.00  & 20.93  & 1.00  & 595.15  \\ \hline
        S306 & 175133.33  & 729.72  & 6.87  & 1.00  & 17.67  & 1.00  & 578.68  \\ \hline
        S356 & 180981.94  & 754.09  & 8.47  & 1.00  & 113.03  & 0.95  & 575.90  \\ \hline
        S352 & 148385.00  & 618.27  & 7.18  & 1.00  & 0.03  & 0.98  & 485.75  \\ \hline
        S194 & 140784.72  & 586.60  & 5.89  & 1.00  & 12.20  & 1.00  & 468.09  \\ \hline
        S143 & 137978.33  & 574.91  & 8.11  & 1.00  & 41.14  & 0.76  & 385.08  \\ \hline
        S247 & 78747.22  & 328.11  & 5.09  & 1.00  & 13.34  & 1.00  & 267.13  \\ \hline
        S031 & 62434.85  & 260.15  & 5.04  & 1.00  & 0.10  & 0.96  & 208.50  \\ \hline
        S365 & 57820.83  & 240.92  & 6.03  & 1.00  & 31.23  & 0.99  & 197.62  \\ \hline
        S284 & 64718.06  & 269.66  & 7.34  & 1.00  & 9.17  & 0.76  & 186.28  \\ \hline
        S040 & 48340.91  & 201.42  & 6.23  & 1.00  & 24.51  & 0.78  & 142.42  \\ \hline
        S367 & 39901.52  & 166.26  & 6.07  & 1.00  & 77.60  & 0.49  & 92.34  \\ \hline
        S374 & 68366.67  & 284.86  & 10.81  & 1.00  & 6.07  & 0.02  & 80.34  \\ \hline
        S364 & 43580.30  & 181.58  & 6.66  & 1.00  & 0.04  & 0.27  & 78.03  \\ \hline
        S346 & 35212.12  & 146.72  & 4.19  & 1.00  & 88.37  & 0.32  & 67.92  \\ \hline
        S292 & 15266.67  & 169.63  & 6.04  & 0.49  & 0.04  & 0.17  & 59.93  \\ \hline
        S080 & 26718.06  & 111.33  & 5.85  & 1.00  & 23.14  & 0.41  & 57.98  \\ \hline
        S055 & 36425.76  & 151.77  & 7.58  & 1.00  & 101.16  & 0.05  & 46.63  \\ \hline
        S078 & 14255.00  & 141.14  & 6.18  & 0.45  & 0.15  & 0.14  & 43.81  \\ \hline
        S244 & 22786.11  & 94.94  & 5.21  & 1.00  & 23.64  & 0.18  & 37.37  \\ \hline
        S218 & 21504.17  & 89.60  & 5.21  & 1.00  & 26.77  & 0.18  & 35.62  \\ \hline
        S003 & 18247.22  & 95.54  & 6.27  & 0.83  & 0.20  & 0.19  & 35.38  \\ \hline
        S294 & 26169.44  & 109.04  & 3.84  & 1.00  & 13.85  & 0.04  & 33.65  \\ \hline
        S086 & 24929.17  & 121.02  & 7.96  & 0.87  & 0.86  & 0.08  & 30.73  \\ \hline
        S189 & 14820.00  & 98.15  & 5.11  & 0.66  & 0.00  & 0.01  & 27.97  \\ \hline
        S007 & 11580.00  & 48.25  & 4.93  & 1.00  & 28.46  & 0.07  & 16.84  \\ \hline
        S266 & 10865.00  & 45.27  & 2.10  & 1.00  & 2.51  & 0.00  & 13.92  \\ \hline
        S123 & 10748.33  & 44.78  & 3.48  & 1.00  & 5.06  & 0.00  & 13.90  \\ \hline
        S393 & 50.00  & 50.00  & 1.64  & 0.14  & 0.19  & 0.00  & 9.94  \\ \hline
        S135 & 120.83  & 24.17  & 2.16  & 0.09  & 0.00  & 0.00  & 6.31  \\ \hline
        S190 & 72.73  & 24.24  & 1.58  & 0.09  & 0.00  & 0.00  & 6.27  \\ \hline
        S394 & 240.00  & 24.00  & 2.02  & 0.28  & 0.01  & 0.00  & 6.22  \\ \hline
        S062 & 125.00  & 25.00  & 1.97  & 0.07  & 0.04  & 0.00  & 5.84  \\ \hline
        S321 & 50.00  & 25.00  & 1.33  & 0.05  & 0.07  & 0.00  & 5.59  \\ \hline
        S334 & 205.56  & 20.56  & 2.01  & 0.30  & 0.01  & 0.00  & 5.29  \\ \hline
        S150 & 3010.00  & 12.54  & 5.22  & 1.00  & 15.81  & 0.00  & 4.54  \\ \hline
        S240 & 74.24  & 14.85  & 1.47  & 0.27  & 0.00  & 0.00  & 3.92  \\ \hline
        S254 & 268.33  & 10.32  & 2.10  & 0.19  & 0.00  & 0.00  & 2.97  \\ \hline
        S251 & 60.00  & 10.00  & 1.19  & 0.23  & 0.01  & 0.00  & 2.60  \\ \hline
\end{longtable}

\cref{tab:ping}显示了最优的50家供应商的指标以及最终评分。
\subsection{问题二的模型}
本问中分为了三个小问,经过分析我们发现这三个小问都属于规划问题,其所用的约束条件是互通的。

(1)第一小问是在前面的基础上选出尽量少的供应商来保障该企业的生产;
(2)第二小问是求出未来24周企业最经济的原材料采购方案,包括选取的供应商以及订货量;
(3)第三小问要求出损耗最小的转运方案,即给确定的供应商选择恰当的转运公司。

\subsubsection{模型建立}
\leftline{\textbf{(一)约束条件的建立}}

设企业在第$w$周向某供应商$s$的订货量为$D_{w,s}$,第$w$周某供应商$s$的供货量为$G_{w,s}$,二者关系见\cref{eq:GD}。

\begin{equation}
G_{w,s}=D_{w,s}+N_s(\mu_s,\sigma_s^2)
\label{eq:GD}
\end{equation}

矩阵
$
\mathbf{X_w} = \left(
    \begin{array}{cccc}
    x_{11} & x_{12} & \ldots & x_{1s}\\
    x_{21} & x_{22} & \ldots & x_{2s}\\
    \vdots & \vdots & \ddots & \vdots\\
    x_{t1} & x_{t2} & \ldots & x_{ts}\\
    \end{array} \right)
$表示第$w$周转运商的选取情况。
其中
$
x_{ij} =
    \begin{cases}
        0 &  \text{未被选择} ,\\
        1 &  x \text{被选择} .
    \end{cases}
$,表示$w$周选转运商$t$为供应商$s$运货,$t\in [1,8]$,$s\in [1,402]$。

%TODO jianjv
由模型假设,一家供货商仅由一家转运商运货,则$\forall s$,都有\cref{eq:re_j1}的约束。

\begin{equation}
\sum\limits_t X_w(t,s)\le 1
\label{eq:re_j1}
\end{equation}

又因为一家转运商的转运能力为6000$m^3$/周,所以$\forall t$,都有\cref{eq:re_j2}的约束。

\begin{equation}
\sum\limits_s X_w(t,s)G_{w,s}\le 6000
\label{eq:re_j2}
\end{equation}

矩阵
$
\mathbf{L} = \left(
    \begin{array}{cccc}
    l_{11} & l_{12} & \ldots & l_{1w}\\
    l_{21} & l_{22} & \ldots & l_{2w}\\
    \vdots & \vdots & \ddots & \vdots\\
    l_{t1} & l_{t2} & \ldots & l_{tw}\\
    \end{array} \right)
$表示转运公司损失率情况,转运公司$t$在第$w$周的损失率为$L(t,w)~N(\mu,\sigma^2)$。

企业在第$w$周的总收货量为$R_(A,w)$,$R_(B,w)$,$R_(C,w)$,$R_(A,w)$的公式如\cref{eq:RAw}所示。
%TODO 2
\begin{equation}
R_(A,w)=\sum\limits_{S_a} G_(w,s)(1-X_w(:,S_a)^T\cdot L(:,w))
\label{eq:RAw}
\end{equation}

同理可得$R_(B,w)$,$R_(C,w)$。

由模型假设,企业库存要保证2周的生产需求,设第$w$周生产剩余$K_w$的原料,则有\cref{eq:Kw}的约束。

\begin{equation}
K_w=K_{w-1}+\frac{R_{A,w}}{0.6}+\frac{R_{B,w}}{0.66}+\frac{R_{C,w}}{0.72}\ge 2.82\times 10^4\times 2
\label{eq:Kw}
\end{equation}

另外对于供货量有\cref{eq:Gmax}的约束。

\begin{equation}
G_{w,s}<\widetilde{G}_{w,s}
\label{eq:Gmax}
\end{equation}

$\widetilde{G}_{w,s}$为预测所得到的第$w$周第$s$家供货商的供货量。通过绘制供应商近5年的供货量的波动图可以发现,绝大部分公司的供货量有明显的季节性与周期性,因此可以使用时间序列分析中的Holt-Winters方法对未来24周的供货量进行合理预测。
\leftline{\textbf{(二)目标函数的建立}}

第一小问的目标函数为\cref{eq:yue1}。

\begin{equation}
min \sum\limits_{w=1}^24 G_{w,s\neq 0}
\label{eq:yue1}
\end{equation}

第一小问总的数学模型为:
\[
min \quad \sum\limits_{w=1}^{24} G_{w,s\neq 0}
\]

$$ s.t. \quad \left\{
\begin{aligned}
    & K_w=K_{w-1}+\frac{R_{A,w}}{0.6}+\frac{R_{B,w}}{0.66}+\frac{R_{C,w}}{0.72}\ge 2.82\times 10^4\times 2 \\
    & G_{w,s}<\widetilde{G}_{w,s}
\end{aligned}
\right.
$$

第二小问的目标函数为\cref{eq:yue2}。
\begin{equation}
min \quad \sum\limits_w(1.2G_{A,w}+1.1G_{B,w}+G_{C,w})
\label{eq:yue2}
\end{equation}
第二小问总的数学模型为:
\[
min \quad \sum\limits_w(1.2G_{A,w}+1.1G_{B,w}+G_{C,w})
\]

$$ s.t. \quad \left\{
\begin{aligned}
    & K_w=K_{w-1}+\frac{R_{A,w}}{0.6}+\frac{R_{B,w}}{0.66}+\frac{R_{C,w}}{0.72}\ge 2.82\times 10^4\times 2 \\
    & G_{w,s}<\widetilde{G}_{w,s}
\end{aligned}
\right.
$$

第三小问的目标函数为\cref{eq:yue3}
\begin{equation}
min \quad (\frac{G_{A,w}-R_{A,w}}{0.6}+\frac{G_{B,w}-R_{B,w}}{0.66}+\frac{G_{C,w}-R_{C,w}}{0.72})
\label{eq:yue3}
\end{equation}

第三小问总的数学模型为:
\[
min \quad (\frac{G_{A,w}-R_{A,w}}{0.6}+\frac{G_{B,w}-R_{B,w}}{0.66}+\frac{G_{C,w}-R_{C,w}}{0.72})
\]

$$ s.t. \quad \left\{
\begin{aligned}
    & \sum\limits_t X_w(t,s)\le 1 \\
    & \sum\limits_s X_w(t,s)G_{w,s}\le 6000
\end{aligned}
\right.
$$
\subsubsection{模型求解}
(第一小问)目标函数:供应商的数量最少。约束条件:企业每周的产能至少为 28200 ,第一周需要满足两周的产能。使用python的pulp库进行整数规划,共有402x24个变量,代表402个企业每周供应或者不供应,供应为1,不供应为0,因此目标函数遍历整个矩阵,计算1的数量,使1的数量最少。通过8家转运商的数据,计算均值,求出最大的损失比例,用1减去这个比例,即为剩余比率,402家企业24周的数据乘剩余比例进行近似估计,在下面的第三问和第四问将使用更多准备的损失率估计方法。产能的约束条件为是否供货的0-1矩阵与预测出的产能矩阵切片相乘,使其满足企业每周的产能。

(第二小问)目标函数:最经济的原料订购方案,使得订货量的值最少。约束条件:A类采购成本比C高百分之二十,B比C高百分之十,以C为基准,使得原材料订购最少。每家企业每周订购的原材料大于等于0,小于基于时间序列模型预测的值。使用python pulp库,计算8家转运商的均值和方差,由于均值和方差生成正态分布的随机数,因为有一个随机的过程,结果会上下浮动。通过对excel表格的处理,将ABC三种原料的供货商分开处理,在计算订购原料数值时候加入采购成本比例,作为目标函数,经过数学规划的整数规划,进行优化,使其得到最优解。在计算约束条件的时候,为达到每周产能,加入ABC三种原料的消耗比例。

(第三小问)目标函数:转运损耗最小。约束条件:一家供货商仅由一家转运商转运,每家转运商的运输能力为每周 6000 立方米。需要计算24周8家转运商对前面小问求出的270家供应商的转运情况,涉及到的参数庞大,因此分24周进行处理,每次计算1周,在每周采用矩阵的形式进行计算。该问同样是数学规划中的0-1规划,首先需要计算出8家转运商对270家供应商的转运情况,具体的约束为一家供货商仅由一家公司转运,如若转运,则表示供货商与转运商的关系的矩阵中相应位置为1,否则为0,按列所求和小于等于1.通过之前求得得供货商的0-1矩阵和供货量的预测值,计算出24周每周供货商家所能供货的总和。有供应才有转运,在考虑通过正态分布计算出损失值
\subsection{问题三的模型}
问题三有两小问,第一小问求采购方案,第二小问求转运方案,但是把原材料A的优先级变高,把C的优先级变低了。
\subsubsection{模型建立}

\leftline{\textbf{(二)目标函数的建立}}
问题三第一小问的目标函数为:。

\begin{equation}
min \quad \sum\limits_w(R_{A,w}+R_{B,w}+R_{C,w})
\label{eq:yue4}
\end{equation}

问题三第一小问总的数学模型为:
\[
min \quad \sum\limits_w(R_{A,w}+R_{B,w}+R_{C,w})
\]

$$ s.t. \quad \left\{
\begin{aligned}
    & K_w=K_{w-1}+\frac{R_{A,w}}{0.6}+\frac{R_{B,w}}{0.66}+\frac{R_{C,w}}{0.72}\ge 2.82\times 10^4\times 2 \\
    & G_{w,s}<\widetilde{G}_{w,s}
\end{aligned}
\right.
$$

问题三第二小问的目标函数为\cite{eq:yue5}。

\begin{equation}
min \quad (\frac{G_{A,w}-R_{A,w}}{0.6}+\frac{G_{B,w}-R_{B,w}}{0.66}+\frac{G_{C,w}-R_{C,w}}{0.72})
\label{eq:yue5}
\end{equation}

问题三第二小问总的数学模型为:
\[
min \quad (\frac{G_{A,w}-R_{A,w}}{0.6}+\frac{G_{B,w}-R_{B,w}}{0.66}+\frac{G_{C,w}-R_{C,w}}{0.72})
\]

$$ s.t. \quad \left\{
\begin{aligned}
    & \sum\limits_t X_w(t,s)\le 1 \\
    & \sum\limits_s X_w(t,s)G_{w,s}\le 6000
\end{aligned}
\right.
$$


\subsection{问题四的模型}
\subsubsection{模型建立}
\leftline{\textbf{(一)约束条件的建立}}
经分析得问题四的约束条件为\cite{eq:re_j1},\cite{eq:re_j2}。

\leftline{\textbf{(二)目标函数的建立}}

问题四的目标函数为\cite{eq:yue5}。

问题四总的数学模型为:
\[
min \quad (\frac{G_{A,w}-R_{A,w}}{0.6}+\frac{G_{B,w}-R_{B,w}}{0.66}+\frac{G_{C,w}-R_{C,w}}{0.72})
\]

$$ s.t. \quad \left\{
\begin{aligned}
    & \sum\limits_t X_w(t,s)\le 1 \\
    & \sum\limits_s X_w(t,s)G_{w,s}\le 6000
\end{aligned}
\right.
$$

\section{模型评价与改进}

\subsection{模型优点}
\begin{itemize}
    \item 本题中数个模型均采用数学规划模型,模型约束严格,满足题设
    \item 模型中采用随机变量来模拟实际情况,使得模型有更好的适应性
\end{itemize}
\subsection{模型缺点}
\begin{itemize}
    \item 模型中未知参数太多,导致模型求解很慢。
    \item 模型中使用的随机变量,导致有时会有冲突的约束条件,使得模型无解。
\end{itemize}
\subsection{模型改进}
\begin{itemize}
    \item 简化模型,数万个参数常常超出程序的处理能力
    \item 约束随机变量,使得其不出现出图的约束条件
\end{itemize}



% TODO 参考文献, 注意: 参考文献格式很重要! 参考文献必须列出且在正文中标注(\cite{})!!, 不得直接抄参考文献原文!!!
%参考文献
\begin{thebibliography}{9}%宽度9
 \bibitem{bib:1} 刘勇. 原材料采购管理模式研究[D].重庆大学,2010.
 \bibitem{bib:2} Dickson G W.An Analysis of Vendor Selection Systems and Decisions[J].Journal of Purchasing,1966,2(1):5-17.
 \bibitem{bib:3} 关志民. 供应链环境下供应商选择方法及其应用研究[D].东北大学,2006.
\end{thebibliography}




\end{document} 