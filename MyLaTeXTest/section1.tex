\documentclass[UTF8]{ctexart}
\usepackage{amsmath}
\usepackage{graphicx}
\usepackage{ctex}
\usepackage{graphicx}
\usepackage{caption2}
\usepackage{subfigure}
\usepackage{float}
\usepackage{listings}
\usepackage{fancyhdr}
\CTEXsetup[format={\Large\bfseries}]{section}
\title{\vspace{-3cm}\textbf{\zihao{1}论“能战方能止战”\vspace{+10cm}}
\pagestyle{fancy}
\fancyhead{}
\renewcommand\headrulewidth{0pt}
\begin{document}
\date{}
\maketitle
\section{开场白}
.........................
\section{文化与历史沿革}
\subsection{}
现在我们看一下我国的版图,向北是辽远的蒙古高原与西伯利亚平原,在东南有大洋阻隔,
而西面又是号称世界屋脊的青藏高原和帕米尔高原,同时长江黄河流域水草丰美,这样的地理条件
孕育出古老灿烂绵延数千年不绝的华夏文明,同时也阻隔了我们对外扩张的野望,造就了我们民族
在整体上爱好和平的性格。
尽管如此,在我们“几千寒热”的历史中,仍然免不了“上疆场彼此弯弓月。流遍了,
郊原血。”在近代以前,我们或是与游牧民族打交道,或是王朝更迭的内部战争,对于战争
的反思与总结也从未停止。在一开始,我们主张的是“兵者不祥之器,非君子之器,不得已而用之”,
主张的是“安得猛士兮守四方”,
\end{document}